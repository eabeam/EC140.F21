\documentclass[11pt]{article}




\usepackage[sfdefault]{FiraSans} %% option 'sfdefault' activates Fira Sans as the default text font
\usepackage[T1]{fontenc}
\renewcommand*\oldstylenums[1]{{\firaoldstyle #1}}

\usepackage{natbib}
\usepackage[french,english]{babel}
\usepackage{numprint}
\usepackage{multirow}
\usepackage{rotating}
\usepackage{fancyhdr}
\usepackage{booktabs}
\usepackage{multicol}
\usepackage{hyperref}\hypersetup{colorlinks=true}

\usepackage{amsmath,amssymb,amsfonts,textcomp}
\usepackage{color}
\usepackage{calc}
 \setlength{\tabcolsep}{8pt}
\usepackage{setspace}
\onehalfspacing
\usepackage{longtable}
\usepackage{graphicx}
\usepackage[margin=1in]{geometry}
\setlength{\parindent}{0pt}
\usepackage[bottom]{footmisc}
\pagestyle{fancy}
\usepackage{titlesec}
\usepackage{lipsum}
\usepackage{cancel}
\usepackage{multicol}

\usepackage{amsmath,amssymb}
\usepackage{lmodern}
\usepackage{iftex}
\ifPDFTeX
  \usepackage[T1]{fontenc}
  \usepackage[utf8]{inputenc}
  \usepackage{textcomp} % provide euro and other symbols
\else % if luatex or xetex
  \usepackage{unicode-math}
  \defaultfontfeatures{Scale=MatchLowercase}
  \defaultfontfeatures[\rmfamily]{Ligatures=TeX,Scale=1}
\fi

\titleformat{\section}
  {\normalfont\Large\scshape\bfseries}{\thesection}{1em}{}
  \titlespacing{\section}{0pt}{10pt}{0pt}
\titleformat{\subsection}
  {\normalfont\bfseries}{\thesection}{1em}{}
  \titlespacing{\subsection}{0pt}{6pt}{0pt}
\providecommand{\tightlist}{%
  \setlength{\itemsep}{0pt}\setlength{\parskip}{0pt}}\newenvironment{itemize*}%

  
\lhead{EC140 - Economic Development}
\rhead{Version: \today}
\setlength\parskip{0.10in}
\begin{document}
\thispagestyle{plain}
\singlespacing


Version: \today \hfill Fall 2021\\
EC140: Economic Development
%\vspace{1.5cm}
\begin{center}
\Large{\textbf{Capstone 1: Policy Memo}}\\
\end{center}
\bigskip

%
%\hypertarget{ec140-economic-development}{%
%\section{EC140 Economic Development}\label{ec140-economic-development}}
%%
%{[}TOC{]}
%
%\hypertarget{learning-objectives}{%
%\subsection{Learning Objectives}\label{learning-objectives}}
%
%\begin{longtable}[]{@{}
%  >{\raggedright\arraybackslash}p{(\columnwidth - 2\tabcolsep) * \real{0.10}}
%  >{\raggedright\arraybackslash}p{(\columnwidth - 2\tabcolsep) * \real{0.90}}@{}}
%\toprule
%& \\
%\midrule
%\endhead
%\textbf{LO1} & To synthesize a range of voices and perspectives and
%develop a broad understanding of the determinants of poverty and
%economic growth. \\
%\textbf{LO2} & To use economic theory to understand barriers to economic
%growth, challenges and trade-offs that individuals and governments in
%low and middle-income countries face, and the effectiveness of small and
%large-scale interventions to reduce poverty. \\
%\textbf{LO3} & To understand how economists generate rigorous knowledge,
%to interpret quan- titative results, and to critically examine
%strategies and assumptions underlying measurement of causal impacts. \\
%\textbf{LO4} & To communicate economic issues clearly and concisely in
%multiple forms of writing. \\
%\bottomrule
%\end{longtable}
%
%\hypertarget{syllabus}{%
%\subsection{Syllabus}\label{syllabus}}
%
%\href{files/ec140_syllabus_f21.pdf}{Fall 2020 syllabus}
%
%\hypertarget{grading}{%
%\subsection{Grading}\label{grading}}
%
%\begin{longtable}[]{@{}ll@{}}
%\toprule
%Activity & \% \\
%\midrule
%\endhead
%Reading responses & 20\% \\
%Class participation & 20\% \\
%Capstone assignments & 60\% \\
%\bottomrule
%\end{longtable}
%
%\hypertarget{capstone-1-policy-memo}{%
%\section{Capstone 1: Policy Memo}\label{capstone-1-policy-memo}}

\hypertarget{learning-objectives-1}{%
\section*{Learning Objectives}\label{learning-objectives-1}}

\begin{enumerate}
\def\labelenumi{\arabic{enumi}.}
\tightlist
\item
  Demonstrate understanding of key topics in Unit 1

  \begin{enumerate}
  \def\labelenumii{\alph{enumii}.}
  \tightlist
  \item
    Defining poverty and considering the ``economic lives of the poor''
  \item
    The role of geography, institutions, and foreign aid in economic
    growth and the economic well-being of citizens today
  \end{enumerate}
\item
  Comunicate ideas in a concise format through a policy memo
\end{enumerate}

\hypertarget{prompt}{%
\section*{Prompt}\label{prompt}}

Your assignment is to choose a low- or middle income country (as defined
by the
\href{https://datahelpdesk.worldbank.org/knowledgebase/articles/906519}{World
Bank}) and write a brief policy ``briefing'' memo directed at a new
program analyst at USAID.

The following countries are \textbf{recommended} because profiles of
individual families/households are available through one of the sources
below: Bangaldesh, Bolivia, Bulgaria, Brazil, Burundi , Burkina Faso,
Cambodia, China, Cote d'Ivoire, Colombia, Cameroon, Egypt, Ethiopia,
Ghana, Guatemala, Haiti, Iran, Indonesia, India, Jordan, Kyrgyzstan,
Kazakhstan, Kenya, Lebanon, Liberia, Myanmar, Malawi, Mongolia, Mexico,
Mozambique, Nigeria, Nepal, Pakistan, Peru, Philippines, Papua New
Guinea, Russia, Romania, Rwanda, Somalia, Switzerland, South Africa,
Serbia, Sri Lanka, Tunisia, Tanzania, Thailand, Togo, Turkey, Ukraine,
Vietnam, Zimbabwe

There are \textbf{four} key elements your memo should include:

\begin{enumerate}
\def\labelenumi{\arabic{enumi}.}
\tightlist
\item
  \textbf{Economic status:} Description of the economic status of the
  country based on recent data. You should pull this information form a
  source such as the \href{https://data.worldbank.org/}{World Bank},
  \href{http://hdr.undp.org/en/content/download-data}{UN Human
  Development Reports}, or other large-scale surveys and data collection
  efforts (\href{https://www.dhsprogram.com/}{Demographic and Health
  Surveys},
  \href{https://www.worldbank.org/en/programs/lsms/initiatives/lsms-plus}{LSMS},
  etc.). This discussion should include the following factors

  \begin{enumerate}
  \def\labelenumii{\arabic{enumii}.}
  \tightlist
  \item
    Current GDP/capita levels and recent trends
    \item A measure of poverty and recent trends
  \item
    At least one additional that reflects quality of life and recent
    trends in them
  \end{enumerate}
\item
  \textbf{Institutions:} Description of key events and/or
  characteristics of the country that has influenced its economic
  development (that you described above) and a summary of its political
  and economic institutions. Make sure your discussion is supported w/
  evidence.

  \begin{enumerate}
  \def\labelenumii{\arabic{enumii}.}
  \tightlist
  \item
    You'll need to cite our readings (where relevant) in this section
  \item
    You will also need to bring in \textbf{at least 2 external
    (non-class) sources}
  \end{enumerate}
\item
  \textbf{Foreign aid and effectiveness} Summary of aid this country has
  received and is currently receiving from
  \href{https://www.usaid.gov/}{USAID} (and/or other places) and
  discussion of whether this aid has been effective

  \begin{enumerate}
  \def\labelenumii{\arabic{enumii}.}
  \tightlist
  \item
    Effectiveness can be based on your own impressions, but it should
    cite other researchers/authors. This could include our class
    readings, but it also should include \textbf{at least 2 external
    sources}.
  \end{enumerate}
\item
  \textbf{Citizen profile:} To help our analyst get a sense of the lives
  of actual citizens of this country, include a ``pop-out'' box that
  profiles one individual or one family. This information can be drawn
  from Gapminder's \href{https://www.gapminder.org/dollar-street}{Dollar
  Street},
  \href{http://www.portfoliosofthepoor.com/portfolios.asp}{Portfolios of
  the Poor} (applicable for Bangladesh, India, South Africa),
  \href{https://www.fsdkenya.org/}{Kenyan Financial Diaries},
  \href{https://bfaglobal.com/our-work/financial-diaries/}{BFA Global
  Financial Diaries} (China, India, Mexico, South Africa, Kenya,
  Mozambique, Pakistan, Rwanda, Tanzania, Uganda)
\end{enumerate}

Then, include a separate page that has the following:

\begin{itemize}
\tightlist
\item
  Definintions of all terms used (GDP/capita, PPP, HDI, etc.)

  \begin{itemize}
  \tightlist
  \item
    Make sure it's clear in your terms what they mean \emph{exactly}.
    How is poverty rate defined? Is GDP/captia in PPP dollars, US
    dollars? Based to what year?
  \end{itemize}
\item
  References list

  \begin{itemize}
  \tightlist
  \item
    Formatted bibliography using
    \href{https://owl.purdue.edu/owl/research_and_citation/resources.html}{APA,
    MLA, or Chicago}.
  \item
    Make sure to include proper citation for any course readings
    readings \emph{and} data sources as well as external
    articles/readings.
  \end{itemize}
\end{itemize}

\hypertarget{specifications}{%
\section*{Specifications}\label{specifications}}

\begin{itemize}
\item
  1200-1800 words (excluding references and definitions). You may want
  to consider bulleted lists, tables, etc. to stay within the word
  count.
\item
  Descriptive title (ie. not ``TONGA BRIEF'')
\item
  Spend some time formatting your brief to make it look nice!
\item
  You'll need quite a few outside sources. I expect the following:

  \begin{itemize}
  \tightlist
  \item
    Data sources for economic profile
  \item
    Academic and non-academic literature for institutions and foreign
    aid effectiveness
  \item
    Data sources for current aid
  \item
    External source for citizen profile
  \end{itemize}
\item
  When bringing in outside sources, use in-text citations, be brief, and avoid large
  quotation blocks

  \begin{itemize}
  \tightlist
  \item
    Good: ``Botswana's standard of living has enabled the government to
    invest more in education, a key''engine of economic prosperity"
    (Acemoglu and Robinson 2009, p77)
  \item
    No: ``Inclusive economic instituions pave the way for two other
    engines of economic prosperity: technology and education.''
    (Acemoglu and Robinson 2009, p77) \emph{Instead, link to your own
    points and pare down.}
  \item
    No: ``In their book, \emph{Why Nations Fail,} Acemoglu and Robinson
    (p77) highlight the importance of technology and education as
    engines of economic prosperity.'' \emph{We don't need the book
    title, and make the link to what you're saying clearer. While we can
    state their arguments, it's better to integrate them with our own.}
  \item
    No: ``Botswana has invested heavily in education, as inclusive
    economic instituions pave the way for two other engines of economic
    prosperity: technology and education.'' Well, now you've plagarised.
    Even if you include AR in the references!
  \end{itemize}
\end{itemize}

\hypertarget{deliverables}{%
\section*{Deliverables}\label{deliverables}}

\hypertarget{rough-draft-september-20}{%
\subsection*{Rough draft (Sept
20 at 1:10pm)}\label{rough-draft-september-20}}


During class, you'll peer review each other's drafts.

What counts as ``complete enough'' for a rough draft:

\begin{itemize}
\tightlist
\item
  Selected an eligible country
\item
  Content in all four key elements

  \begin{itemize}
  \tightlist
  \item
    Data and indicators in place for part (1)
  \item
    External sources (2+) brought in for key events/characteristics
  \end{itemize}
\item
  Definition of terms
\end{itemize}

\hypertarget{final-draft-september-27}{%
\subsection*{Final draft (Sept.
27 at 1:10pm)}\label{final-draft-september-27}}


We will have a remote activity to work with the drafts.
%
%\hypertarget{capstone-2-mini-grant-proposal}{%
%\subsection{Capstone 2: Mini-grant
%proposal}\label{capstone-2-mini-grant-proposal}}
%
%\hypertarget{capstone-3-policy-brief}{%
%\subsection{Capstone 3: Policy brief}\label{capstone-3-policy-brief}}

\end{document}
