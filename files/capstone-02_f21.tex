\documentclass[11pt]{article}




\usepackage[sfdefault]{FiraSans} %% option 'sfdefault' activates Fira Sans as the default text font
\usepackage[T1]{fontenc}
\renewcommand*\oldstylenums[1]{{\firaoldstyle #1}}

\usepackage{natbib}
\usepackage[french,english]{babel}
\usepackage{numprint}
\usepackage{multirow}
\usepackage{rotating}
\usepackage{fancyhdr}
\usepackage{booktabs}
\usepackage{multicol}
\usepackage{hyperref}\hypersetup{colorlinks=true}

\usepackage{amsmath,amssymb,amsfonts,textcomp}
\usepackage{color}
\usepackage{calc}
 \setlength{\tabcolsep}{8pt}
\usepackage{setspace}
\onehalfspacing
\usepackage{longtable}
\usepackage{graphicx}
\usepackage[margin=1in]{geometry}
\setlength{\parindent}{0pt}
\usepackage[bottom]{footmisc}
\pagestyle{fancy}
\usepackage{titlesec}
\usepackage{lipsum}
\usepackage{cancel}
\usepackage{multicol}

\usepackage{amsmath,amssymb}
\usepackage{lmodern}
\usepackage{iftex}
\ifPDFTeX
  \usepackage[T1]{fontenc}
  \usepackage[utf8]{inputenc}
  \usepackage{textcomp} % provide euro and other symbols
\else % if luatex or xetex
  \usepackage{unicode-math}
  \defaultfontfeatures{Scale=MatchLowercase}
  \defaultfontfeatures[\rmfamily]{Ligatures=TeX,Scale=1}
\fi

\titleformat{\section}
  {\normalfont\Large\scshape\bfseries}{\thesection}{1em}{}
  \titlespacing{\section}{0pt}{10pt}{0pt}
\titleformat{\subsection}
  {\normalfont\bfseries}{\thesection}{1em}{}
  \titlespacing{\subsection}{0pt}{6pt}{0pt}
\providecommand{\tightlist}{%
  \setlength{\itemsep}{0pt}\setlength{\parskip}{0pt}}\newenvironment{itemize*}%

  
\lhead{EC140 - Economic Development}
\rhead{Version: \today}
\setlength\parskip{0.10in}
\begin{document}
\thispagestyle{plain}
\singlespacing


Version: \today \hfill Fall 2021\\
EC140: Economic Development
%\vspace{1.5cm}
\begin{center}
\Large{\textbf{Capstone 2: RCT Proposal (mini-grant)}}\\
\end{center}
\bigskip


\hypertarget{learning-objectives-1}{%
\section*{Learning objectives}\label{learning-objectives-1}}

The objective of this capstone is to provide an opportunity to connect
the in-class topics with real-world challenges, to strengthen your
skills in communicating clearly and concisely about an issue, and to
develop your abilities in thinking critically about writing and
providing constructive feedback.

\hypertarget{prompt-1}{%
\section*{Prompt}\label{prompt-1}}

Throughout this class we've discussed a lot of problems facing the
world's poor and potential solutions. The Global Innovation Fund
provides up to
\href{https://www.globalinnovation.fund/apply/about/what-we-fund/}{\$230,000
in funding} for pilot projects that provide ``social innovations that
aim to improve the lives and opportunities of millions of people in the
developing world.'' Your assignment is to write an adapted mini-grant
proposal to GIF. Pick a problem of interest to you, think about one
possible ``innovation'' that is testable, and think about how you would
evaluate it.

GIF considers four criteria in its funding decisions\footnote{The fifth
  criteria is team composition, but we'll skip that for now}
\begin{enumerate}
\item \textbf{Target market}
\begin{itemize}
\tightlist
\item
  Innovation brings benefits for those who live on the equivalent of \$5
  per day, with extra interest in solutions that affect those living on
  \$2 or less per day or other vulnerable or disadvantaged groups
\end{itemize}

\item \textbf{Innovation and impact:} Will your intervention meaningfully improve the
lives of those living on less than \$5 PPP per day?

\begin{itemize}
\tightlist
\item
  Does your innovation have the potential to deliver substantially
  greater results per dollar than standard development practice?
\item
  What is your evidence of impact to back this claim?

  \begin{itemize}
  \tightlist
  \item
    As these are pilot funds, ``We value any relevant evidence or
    research findings that demonstrate why the innovation is needed,
    such as evidence of customer demand or interest in the innovation.
    We do not expect that strong evidence already exists to prove the
    value of the innovation, but we do need a clear rationale for why
    the innovation could have a greater impact or be more cost effective
    than existing approaches.''
  \end{itemize}
\item
  Will your current request substantially strengthen or test your case
  for impact?
\item
  Does your innovation test, challenge, or improve on business as usual?
\end{itemize}

\item \textbf{Measuring success and lessons learned:} How will you measure your
success? Monitoring and evaluation? Impact evaluation? \emph{For our
class, the correct answer is \textbf{to conduct an impact evaluation}.
Provide details (see below).}

\item \textbf{Potential to scale:} If the pilot is successful, how could the model
scale and grow?

\begin{quote}
GIF only funds innovation which have the potential to scale to benefit
millions of people. Scaling can take place through in any number of
ways-- growth of a successful business or uptake by a government
partner, for example. The \href{https://millionlivesclub.org/}{million
lives club} showcases some of the innovations supported by GIF and
others which are either reaching millions of clients or on their way to
doing so.
\end{quote}

\emph{Note that you don't need to propose anything at scale, just
discuss how one might do it if your intervention was successful}

\end{enumerate}
\hypertarget{format}{%
\section*{Format}\label{format}}

\begin{enumerate}
\def\labelenumi{\arabic{enumi}.}
\item
  Title of project:
\item
  One-sentence pitch \emph{That's right, just one sentence}
\item
  Application summary
\item
  Grant proposal, in one of the following forms

  \begin{enumerate}
  \def\labelenumii{\alph{enumii}.}
  \tightlist
  \item
    2-4 page proposal (\emph{12 point font, single spaced, 1" margins)}

\textbf{OR}

    
    \item 8-15 slide deck + video in which you narrate your slides

  \end{enumerate}
  
  \item
  References
\end{enumerate}


\hypertarget{application-summary}{%
\subsection*{Application summary}\label{application-summary}}

Provide a 1-2 paragraph summary of your application in the space below,
\textbf{maximum 300 words}. Summarize why your approach is innovative
and the expected impacts and potential to scale.

\hypertarget{grant-proposal}{%
\subsection*{Grant proposal}\label{grant-proposal}}

In addition to the application summary, your proposal should have \textbf{four} distinct, labeled sections.



\begin{enumerate}
\def\labelenumi{(\arabic{enumi})}
\item
  Overview of the innovation you propose: What is it, how does it work? 
\item
  Innovation and impact: Demonstrate that your project meets the \textbf{four} criteria above
\item
  Measuring success and lessons learned: Describe a plan to implement a
  randomized impact evaluation to evaluate your innovation. It should
  include the following details:


\begin{enumerate}

\item
  Theory of change: Inputs, Activities, Outputs, Intermediate Outcomes,
  Final outcomes (a diagram is fine)
\item
  Sampling frame: who will be in your sample, how will you recruit them
\item
  Randomization strategy: individual, clustered, explain why
\item
  Treatment arms: how many arms are there, what does each arm (and
  control) receive?
\item
  Outcomes of interest: What are the outcomes you will measure to
  determine if your intervention is effective?
\item
  Measurement strategy: How will you measure these outcomes? Surveys,
  administrative data?
\end{enumerate}
\item
  Potential to scale


\end{enumerate}

As a rough guide, I would expect that section 3 would be about twice as
long as sections 1 and 2, and that section 4 would be very short.

\hypertarget{written-grant-proposal}{%
\paragraph{Written grant proposal}\label{written-grant-proposal}}

Your document should meet the following criteria:

\begin{itemize}
\item
  Include elements described above
\item
  Be 2--4 pages, single spaced
\item
  11 or 12 point reasonable font, 1" margin
\end{itemize}

In order to make this work, you'll note that you won't have much room
for long paragraphs with nice connecting topics sentences. Don't worry
about transitions. Use bullet points when appropriate to make your
points clearer.

\hypertarget{slide-deck-grant-proposal}{%
\paragraph{Slide deck grant proposal}\label{slide-deck-grant-proposal}}

\emph{Note that you still need to prepare the 300-word application
summary if you choose this option!}

Your presentation should contain the following:

\begin{itemize}
\item
  Include all elements described above
\item
  Consist of 8–16 slides
\item
  Slide deck should be attractive, easy to read
\item
  You should record a video of yourself narrating your presentation.
  This is not the same as reading your slides. You can do a screencast
  via Teams or something fancier. I'm just looking for a video of your
  slides w/ an audio component
\item
  Because you are narrating, don't try to cram a written proposal into
  16 slides. It will look awful and be no fun for you. Use bullet points
  to touch on key elements, and then talk us through those points in
  detail
\item
  You don't need to script your narration (though you can) -- but you
  may want to give yourself a few talking points to ensure it goes
  smoothly
\item
  The presentation should be 10–15 minutes.
\end{itemize}

\hypertarget{do-i-need-references}{%
\subsection*{Do I need references?}\label{do-i-need-references}}

Yes, and not just for your grade. Depending on your problem/innovation,
you will need to consult data sources, reports and/or academic articles
for any of these three reasons:

\begin{enumerate}
\def\labelenumi{(\arabic{enumi})}
\item
  To argue why your policy problem matters. It's not sufficient to say
  that ``I believe hunger is a major problem, because when I'm hungry
  it's hard to do anything.'' Rather, ``Malnutrition kills more than 10
  children each year and limits the future of millions more (Yosemite
  and Buggles 2005)''
\item
  To argue why your innovation is innovative. While Smith and Patel
  (2012) have tested the roll-out of microinsurance to protect against
  flooding in Senegal, no one has considered that microinsurance can
  also be adapted to help the poor pay costly veterinary bills."
\item
  To argue why your innovation will work. ``Ho and Gutchow (1988) show
  the promise of inspirational cat posters on children's health and iron
  levels.''
\end{enumerate}

\emph{For grading purposes, you must cite at least 3 peer-reviewed
academic papers, at least two of which must be non-assigned course
readings (bonus pack course readings are fine) .}

\hypertarget{documenting-your-references}{%
\paragraph{Documenting your
references}\label{documenting-your-references}}

Your sources can include readings from class or readings outside of
class, but all should be properly documented in a \textbf{References}
section at the end of the proposal or your slide deck. \emph{The
references section does not count toward the page limit or slide deck
limit.} You can use any standard method of formatting references (APA,
MLA, Chicago), but it should be implemented correctly.

\hypertarget{do-i-need-a-budget}{%
\subsection*{Do I need a budget?}\label{do-i-need-a-budget}}

\textbf{No!}

However, it can be hard to know what \$230k would buy you in terms of an
intervention and evaluation (less than you would think, unfortunately
:no\_mouth: ). And often, governments will provide inputs if your
innovation builds upon an existing program they are interested in. For
example, if you want to add a component to a conditional cash transfer
program, you could assume the cash transfer program is already paid for.
In general, think bigger than distributing flyers to 1000 people but
smaller than a test of universal basic income for the nation of Kenya.

The biggest costs are usually the interventions themselves (ie you give
\$100/person, that adds up!) and staffing needed for surveys. We just
priced out a 1.5 hour 2400-person survey conducted in person in fairly
rural areas in the Philippines at \textasciitilde\$100k (including all
staff costs, transportation, help w/ analysis, etc.).

For this assignment, don't worry about this too much unless you are
\emph{wildly} off (ie could be implemented for \textless{} \$10k or more
than \$5 million).

\hypertarget{i-have-no-ideas}{%
\section*{I have no ideas!}\label{i-have-no-ideas}}

A few approaches:

\begin{enumerate}
\def\labelenumi{\arabic{enumi}.}
\item
  Topic-centric: Pick a general topic that interests you (education,
  sanitation, reproductive health), and dive into the \emph{Poor
  Economics} chapter -- keep an eye out for the papers that catch your
  fancy. Or, if you find someone's work you like, visit their research
  website and see what other stuff they're up to more recently. There
  are lots of summaries of completed an in-progress projects at
  \href{https://www.poverty-action.org/}{IPA's} and
  \href{https://www.povertyactionlab.org/}{JPAL's} websites, and you can
  search by topic
\item
  Recent stuff: Browse \href{https://voxdev.org/}{VoxDev.org} -- it has
  heaps of quick policy briefs on things people are trying.
\item
  GIF: Check out what GIF is
  \href{https://www.globalinnovation.fund/our-investments/\#investments-page+types-of-investments:grant}{funding}
  at the moment
\item
  Dig around for NGOs that work in areas you're interested in.
  \href{https://www.givewell.org/}{GiveWell} has a list of effective
  charities.
\end{enumerate}

You don't need to revolutionize the world by inventing the next big idea
or building a better bed net. But, you can look at what people have
tried or are trying and propose an improvement, an expansion, a new
context, etc. Since it's a pilot, you don't need to know whether it will
work, but you should be able to argue (using evidence) that it
\emph{probably} will make peoples' lives better.

In that vein, you shouldn't take someone's project as is. But, you could
think about what addition/change you might make to someone's
intervention, whether you could apply it in a different context (If
cognitive behavioral therapy helps mothers, could it also help children?
Maybe it would lead to better education outcomes?), etc.

\hypertarget{grading-2}{%
\section*{Grading}\label{grading-2}}

See separate grading rubric. Use it as you work on your proposal.

Note that the rubric will be the same for both methods, with the
exception of the ``Appearance and style'' block. For the written
proposal, that will be graded based on producing a proposal that is
concise, clear, and has minimal grammatical errors. The recorded video
proposal will be graded on the quality of the slides (visually
attractive, concise) and of the narration (makes points clearly, easy to
follow)

\hypertarget{deliverables-1}{%
\subsection*{Deliverables}\label{deliverables-1}}

Your assignment will be submitted in three stages:

\textbf{RCT proposal idea:} Submit on Blackboard by \textbf{October 15
(Friday) @ 1:10pm}

\textbf{RCT proposal draft:} Submit on Blackboard by \textbf{October 29
(Friday) @ 1:10pm}, and we will peer review in class

\textbf{RCT proposal final draft:} Submit on Blackboard by
\textbf{November 7 (Sunday) @ 11:59pm}

You will be graded on your proposal idea (5\%, for completion), draft
brief (10\%, for completion), and your final brief (85\%)



\end{document}
